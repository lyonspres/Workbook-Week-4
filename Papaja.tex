% Options for packages loaded elsewhere
\PassOptionsToPackage{unicode}{hyperref}
\PassOptionsToPackage{hyphens}{url}
%
\documentclass[
  english,
  man]{apa6}
\usepackage{amsmath,amssymb}
\usepackage{lmodern}
\usepackage{ifxetex,ifluatex}
\ifnum 0\ifxetex 1\fi\ifluatex 1\fi=0 % if pdftex
  \usepackage[T1]{fontenc}
  \usepackage[utf8]{inputenc}
  \usepackage{textcomp} % provide euro and other symbols
\else % if luatex or xetex
  \usepackage{unicode-math}
  \defaultfontfeatures{Scale=MatchLowercase}
  \defaultfontfeatures[\rmfamily]{Ligatures=TeX,Scale=1}
\fi
% Use upquote if available, for straight quotes in verbatim environments
\IfFileExists{upquote.sty}{\usepackage{upquote}}{}
\IfFileExists{microtype.sty}{% use microtype if available
  \usepackage[]{microtype}
  \UseMicrotypeSet[protrusion]{basicmath} % disable protrusion for tt fonts
}{}
\makeatletter
\@ifundefined{KOMAClassName}{% if non-KOMA class
  \IfFileExists{parskip.sty}{%
    \usepackage{parskip}
  }{% else
    \setlength{\parindent}{0pt}
    \setlength{\parskip}{6pt plus 2pt minus 1pt}}
}{% if KOMA class
  \KOMAoptions{parskip=half}}
\makeatother
\usepackage{xcolor}
\IfFileExists{xurl.sty}{\usepackage{xurl}}{} % add URL line breaks if available
\IfFileExists{bookmark.sty}{\usepackage{bookmark}}{\usepackage{hyperref}}
\hypersetup{
  pdftitle={The title},
  pdfauthor={First Author1 \& Ernst-August Doelle1,2},
  pdflang={en-EN},
  pdfkeywords={keywords},
  hidelinks,
  pdfcreator={LaTeX via pandoc}}
\urlstyle{same} % disable monospaced font for URLs
\usepackage{graphicx}
\makeatletter
\def\maxwidth{\ifdim\Gin@nat@width>\linewidth\linewidth\else\Gin@nat@width\fi}
\def\maxheight{\ifdim\Gin@nat@height>\textheight\textheight\else\Gin@nat@height\fi}
\makeatother
% Scale images if necessary, so that they will not overflow the page
% margins by default, and it is still possible to overwrite the defaults
% using explicit options in \includegraphics[width, height, ...]{}
\setkeys{Gin}{width=\maxwidth,height=\maxheight,keepaspectratio}
% Set default figure placement to htbp
\makeatletter
\def\fps@figure{htbp}
\makeatother
\setlength{\emergencystretch}{3em} % prevent overfull lines
\providecommand{\tightlist}{%
  \setlength{\itemsep}{0pt}\setlength{\parskip}{0pt}}
\setcounter{secnumdepth}{-\maxdimen} % remove section numbering
% Make \paragraph and \subparagraph free-standing
\ifx\paragraph\undefined\else
  \let\oldparagraph\paragraph
  \renewcommand{\paragraph}[1]{\oldparagraph{#1}\mbox{}}
\fi
\ifx\subparagraph\undefined\else
  \let\oldsubparagraph\subparagraph
  \renewcommand{\subparagraph}[1]{\oldsubparagraph{#1}\mbox{}}
\fi
% Manuscript styling
\usepackage{upgreek}
\captionsetup{font=singlespacing,justification=justified}

% Table formatting
\usepackage{longtable}
\usepackage{lscape}
% \usepackage[counterclockwise]{rotating}   % Landscape page setup for large tables
\usepackage{multirow}		% Table styling
\usepackage{tabularx}		% Control Column width
\usepackage[flushleft]{threeparttable}	% Allows for three part tables with a specified notes section
\usepackage{threeparttablex}            % Lets threeparttable work with longtable

% Create new environments so endfloat can handle them
% \newenvironment{ltable}
%   {\begin{landscape}\begin{center}\begin{threeparttable}}
%   {\end{threeparttable}\end{center}\end{landscape}}
\newenvironment{lltable}{\begin{landscape}\begin{center}\begin{ThreePartTable}}{\end{ThreePartTable}\end{center}\end{landscape}}

% Enables adjusting longtable caption width to table width
% Solution found at http://golatex.de/longtable-mit-caption-so-breit-wie-die-tabelle-t15767.html
\makeatletter
\newcommand\LastLTentrywidth{1em}
\newlength\longtablewidth
\setlength{\longtablewidth}{1in}
\newcommand{\getlongtablewidth}{\begingroup \ifcsname LT@\roman{LT@tables}\endcsname \global\longtablewidth=0pt \renewcommand{\LT@entry}[2]{\global\advance\longtablewidth by ##2\relax\gdef\LastLTentrywidth{##2}}\@nameuse{LT@\roman{LT@tables}} \fi \endgroup}

% \setlength{\parindent}{0.5in}
% \setlength{\parskip}{0pt plus 0pt minus 0pt}

% Overwrite redefinition of paragraph and subparagraph by the default LaTeX template
% See https://github.com/crsh/papaja/issues/292
\makeatletter
\renewcommand{\paragraph}{\@startsection{paragraph}{4}{\parindent}%
  {0\baselineskip \@plus 0.2ex \@minus 0.2ex}%
  {-1em}%
  {\normalfont\normalsize\bfseries\itshape\typesectitle}}

\renewcommand{\subparagraph}[1]{\@startsection{subparagraph}{5}{1em}%
  {0\baselineskip \@plus 0.2ex \@minus 0.2ex}%
  {-\z@\relax}%
  {\normalfont\normalsize\itshape\hspace{\parindent}{#1}\textit{\addperi}}{\relax}}
\makeatother

% \usepackage{etoolbox}
\makeatletter
\patchcmd{\HyOrg@maketitle}
  {\section{\normalfont\normalsize\abstractname}}
  {\section*{\normalfont\normalsize\abstractname}}
  {}{\typeout{Failed to patch abstract.}}
\patchcmd{\HyOrg@maketitle}
  {\section{\protect\normalfont{\@title}}}
  {\section*{\protect\normalfont{\@title}}}
  {}{\typeout{Failed to patch title.}}
\makeatother
\shorttitle{Title}
\keywords{keywords\newline\indent Word count: X}
\DeclareDelayedFloatFlavor{ThreePartTable}{table}
\DeclareDelayedFloatFlavor{lltable}{table}
\DeclareDelayedFloatFlavor*{longtable}{table}
\makeatletter
\renewcommand{\efloat@iwrite}[1]{\immediate\expandafter\protected@write\csname efloat@post#1\endcsname{}}
\makeatother
\usepackage{lineno}

\linenumbers
\usepackage{csquotes}
\ifxetex
  % Load polyglossia as late as possible: uses bidi with RTL langages (e.g. Hebrew, Arabic)
  \usepackage{polyglossia}
  \setmainlanguage[]{english}
\else
  \usepackage[main=english]{babel}
% get rid of language-specific shorthands (see #6817):
\let\LanguageShortHands\languageshorthands
\def\languageshorthands#1{}
\fi
\ifluatex
  \usepackage{selnolig}  % disable illegal ligatures
\fi
\newlength{\cslhangindent}
\setlength{\cslhangindent}{1.5em}
\newlength{\csllabelwidth}
\setlength{\csllabelwidth}{3em}
\newenvironment{CSLReferences}[2] % #1 hanging-ident, #2 entry spacing
 {% don't indent paragraphs
  \setlength{\parindent}{0pt}
  % turn on hanging indent if param 1 is 1
  \ifodd #1 \everypar{\setlength{\hangindent}{\cslhangindent}}\ignorespaces\fi
  % set entry spacing
  \ifnum #2 > 0
  \setlength{\parskip}{#2\baselineskip}
  \fi
 }%
 {}
\usepackage{calc}
\newcommand{\CSLBlock}[1]{#1\hfill\break}
\newcommand{\CSLLeftMargin}[1]{\parbox[t]{\csllabelwidth}{#1}}
\newcommand{\CSLRightInline}[1]{\parbox[t]{\linewidth - \csllabelwidth}{#1}\break}
\newcommand{\CSLIndent}[1]{\hspace{\cslhangindent}#1}

\title{The title}
\author{First Author\textsuperscript{1} \& Ernst-August Doelle\textsuperscript{1,2}}
\date{}


\authornote{

Add complete departmental affiliations for each author here. Each new line herein must be indented, like this line.

Enter author note here.

The authors made the following contributions. First Author: Conceptualization, Writing - Original Draft Preparation, Writing - Review \& Editing; Ernst-August Doelle: Writing - Review \& Editing.

Correspondence concerning this article should be addressed to First Author, Postal address. E-mail: \href{mailto:my@email.com}{\nolinkurl{my@email.com}}

}

\affiliation{\vspace{0.5cm}\textsuperscript{1} Wilhelm-Wundt-University\\\textsuperscript{2} Konstanz Business School}

\abstract{
One or two sentences providing a \textbf{basic introduction} to the field, comprehensible to a scientist in any discipline.

Two to three sentences of \textbf{more detailed background}, comprehensible to scientists in related disciplines.

One sentence clearly stating the \textbf{general problem} being addressed by this particular study.

One sentence summarizing the main result (with the words ``\textbf{here we show}'' or their equivalent).

Two or three sentences explaining what the \textbf{main result} reveals in direct comparison to what was thought to be the case previously, or how the main result adds to previous knowledge.

One or two sentences to put the results into a more \textbf{general context}.

Two or three sentences to provide a \textbf{broader perspective}, readily comprehensible to a scientist in any discipline.
}



\begin{document}
\maketitle

\hypertarget{methods}{%
\section{Methods}\label{methods}}

We report how we determined our sample size, all data exclusions (if any), all manipulations, and all measures in the study.

\hypertarget{participants}{%
\subsection{Participants}\label{participants}}

\hypertarget{material}{%
\subsection{Material}\label{material}}

\hypertarget{procedure}{%
\subsection{Procedure}\label{procedure}}

\hypertarget{data-analysis}{%
\subsection{Data analysis}\label{data-analysis}}

We used R (Version 4.0.4; R Core Team, 2021) and the R-packages \emph{papaja} (Version 0.1.0.9997; Aust \& Barth, 2020), and \emph{tinylabels} (Version 0.2.0; Barth, 2021) for all our analyses.

\hypertarget{import-data}{%
\subsection{Import Data}\label{import-data}}

Question 1
\#\# Integer

\begin{verbatim}
## # A tibble: 4,126 x 6
##    HLTH.SleepHours Hours.Internet Hours.Exercise Hours.Work Hours.News
##              <dbl>          <dbl>          <dbl>      <dbl>      <dbl>
##  1               6             10             14         14          4
##  2               6              5             24          0          4
##  3               6             14              7         35          5
##  4               4             15             10         65          4
##  5               7              2              5         60          1
##  6               7              2              6         50          2
##  7               6              4              1         41          0
##  8               4              4              0         42          0
##  9               7             10              1          0          8
## 10               8             14              7          0          7
## # ... with 4,116 more rows, and 1 more variable: HoursCharity <dbl>
\end{verbatim}

\begin{verbatim}
## tibble [4,126 x 68] (S3: tbl_df/tbl/data.frame)
##  $ Id                         : num [1:4126] 1 1 2 2 3 3 4 4 5 5 ...
##  $ Wave                       : Factor w/ 2 levels "2018","2019": 2 1 2 1 2 1 2 1 2 1 ...
##  $ years                      : num [1:4126] 10.43 9.47 10.61 9.9 10.17 ...
##  $ Age                        : num [1:4126] 47 46 47 46 53 52 60 59 84 84 ...
##  $ Male                       : Factor w/ 2 levels "Male","Not_Male": 1 1 1 1 1 1 2 2 2 2 ...
##  $ Gender                     : num [1:4126] 1 1 1 1 1 1 0 0 0 0 ...
##  $ Edu                        : num [1:4126] 3 3 7 7 4 4 8 7 7 7 ...
##  $ Partner                    : num [1:4126] 1 1 1 1 0 0 1 NA 0 0 ...
##  $ BornNZ                     : num [1:4126] 1 1 1 1 1 1 1 1 1 1 ...
##  $ Employed                   : num [1:4126] 1 0 1 1 1 1 1 NA 0 0 ...
##  $ BigDoms                    : Factor w/ 5 levels "Buddhist","Christian",..: 4 4 4 4 4 4 5 5 2 2 ...
##  $ TSCORE                     : num [1:4126] 3869 3520 3936 3677 3774 ...
##  $ GenCohort                  : Factor w/ 5 levels "Gen Boombers: born >= 1946 & b.< 1961",..: 3 3 3 3 3 3 1 1 2 2 ...
##  $ Religion.Church            : num [1:4126] 0 0 0 0 0 0 2 NA 0 0 ...
##  $ Religion.Believe.Cats      : num [1:4126] 4 4 1 1 1 1 1 NA 3 1 ...
##  $ Relid                      : num [1:4126] 0 0 0 0 0 0 7 7 2 2 ...
##  $ HLTH.Fatigue               : num [1:4126] 2 2 1 2 2 2 1 2 NA 1 ...
##  $ HLTH.SleepHours            : num [1:4126] 6 6 6 4 7 7 6 4 7 8 ...
##  $ HLTH.BMI                   : num [1:4126] 23.1 23.1 35.1 13.1 34 ...
##  $ HLTH.Weight                : num [1:4126] 75 75 120 45 110 110 64 74 54.4 54.4 ...
##  $ HLTH.Height                : num [1:4126] 1.8 1.8 1.85 1.85 1.8 1.8 1.58 1.58 1.6 1.55 ...
##  $ HomeOwner                  : num [1:4126] NA 1 NA 0 NA 1 NA NA NA 0 ...
##  $ Pol.Orient                 : num [1:4126] 3 3 5 3 4 4 3 NA 4 4 ...
##  $ PATRIOT                    : num [1:4126] 4.5 5 6.5 7 4 4 5.5 4 6.5 6 ...
##  $ Env.SatNZEnvironment       : num [1:4126] 7 4 7 7 7 7 4 3 7 8 ...
##  $ Env.MotorwaySpend          : num [1:4126] 5 5 3 5 4 4 4 6 5 6 ...
##  $ Env.PubTransSubs           : num [1:4126] 5 6 5 5 4 4 7 6 4 6 ...
##  $ Env.ClimateChgConcern      : num [1:4126] 6 6 7 7 4 4 6 NA 4 2 ...
##  $ LIFEMEANING                : num [1:4126] 5 6.5 5 4.5 5.5 5.5 7 7 5 6 ...
##  $ Hours.Internet             : num [1:4126] 10 5 14 15 2 2 4 4 10 14 ...
##  $ Issue.GovtSurveillance     : num [1:4126] 3 1 3 3 4 4 1 2 4 3 ...
##  $ Issue.RegulateAI           : num [1:4126] NA 1 NA 4 NA 4 NA 3 NA 4 ...
##  $ Issue.IncomeRedistribution : num [1:4126] 3 2 4 4 2 4 7 6 4 3 ...
##  $ Hours.Exercise             : num [1:4126] 14 24 7 10 5 6 1 0 1 7 ...
##  $ Hours.Work                 : num [1:4126] 14 0 35 65 60 50 41 42 0 0 ...
##  $ Hours.News                 : num [1:4126] 4 4 5 4 1 2 0 0 8 7 ...
##  $ CONSCIENTIOUSNESS          : num [1:4126] 4.75 5.25 5.5 5.5 5 4.25 4.75 4.75 NA 5.25 ...
##  $ EXTRAVERSION               : num [1:4126] 3.25 2.75 4.75 4 3.75 4.5 5.75 4.25 NA 4 ...
##  $ AGREEABLENESS              : num [1:4126] 4.5 5 5 6 5.75 5.25 5 5.25 NA 5 ...
##  $ OPENNESS                   : num [1:4126] 6.5 7 4.25 4.25 6 6 5.75 6.25 NA 5 ...
##  $ Religious                  : Factor w/ 2 levels "Not_Religious",..: 1 1 1 1 1 1 2 2 2 2 ...
##  $ Spiritual.Identification   : num [1:4126] NA 1 NA 5 NA 4 NA NA NA 2 ...
##  $ Believe.God                : Factor w/ 2 levels "Believe God",..: 2 2 1 1 1 1 1 NA 2 1 ...
##  $ Believe.Spirit             : Factor w/ 2 levels "Believe Spirit",..: 2 2 1 1 1 1 1 NA 1 1 ...
##  $ HoursCharity               : num [1:4126] 2 0 0 2 0 0 0 4 0 0 ...
##  $ CharityDonate              : num [1:4126] 180 80 300 100 4200 3500 400 350 50 100 ...
##  $ Your.Personal.Relationships: num [1:4126] 7 6 2 2 8 8 10 10 9 9 ...
##  $ Your.Future.Security       : num [1:4126] 8 10 8 6 8 7 8 7 9 9 ...
##  $ Standard.Living            : num [1:4126] 7 8 8 6 8 8 10 10 9 9 ...
##  $ NZ.Economic.Situation      : num [1:4126] 7 4 2 6 5 6 7 5 7 8 ...
##  $ NZ.Social.Conditions       : num [1:4126] 7 7 2 6 5 5 2 0 9 7 ...
##  $ NZ.Business.Conditions     : num [1:4126] 7 8 2 6 5 5 6 5 9 7 ...
##  $ Emp.JobSecure              : num [1:4126] 7 NA 6 6 5 4 6 NA NA NA ...
##  $ Issue.Food.GMO             : num [1:4126] 1 2 5 5 4 4 7 7 1 4 ...
##  $ Env.SacMade                : logi [1:4126] NA NA NA NA NA NA ...
##  $ KESSLER6sum                : num [1:4126] 5 3 7 7 3 3 0 4 NA 2 ...
##  $ FeelHopeless               : Factor w/ 5 levels "None Of The Time",..: 1 1 2 1 1 1 1 2 NA 1 ...
##  $ FeelDepressed              : Factor w/ 5 levels "None Of The Time",..: 1 1 1 1 1 1 1 1 NA 1 ...
##  $ FeelRestless               : Factor w/ 5 levels "None Of The Time",..: 3 2 4 4 2 2 1 1 NA 2 ...
##  $ EverythingIsEffort         : Factor w/ 5 levels "None Of The Time",..: 2 2 2 3 2 2 1 3 NA 2 ...
##  $ FeelWorthless              : Factor w/ 5 levels "None Of The Time",..: 1 1 1 1 1 1 1 1 NA 1 ...
##  $ FeelNervous                : Factor w/ 5 levels "None Of The Time",..: 3 2 3 3 2 2 1 2 NA 1 ...
##  $ date                       : Date[1:4126], format: "2020-02-02" "2019-02-18" ...
##  $ Hours.Internet_int         : int [1:4126] 10 5 14 15 2 2 4 4 10 14 ...
##  $ Hours.Exercise_int         : int [1:4126] 14 24 7 10 5 6 1 0 1 7 ...
##  $ Hours.Work_int             : int [1:4126] 14 0 35 65 60 50 41 42 0 0 ...
##  $ Hours.News_int             : int [1:4126] 4 4 5 4 1 2 0 0 8 7 ...
##  $ HoursCharity_int           : int [1:4126] 2 0 0 2 0 0 0 4 0 0 ...
\end{verbatim}

Question 2a
\#\# Scale, Center, Transform

\begin{verbatim}
## # A tibble: 6 x 5
##   Pol.Orient   Age Pol.Orient1[,1] Pol.Orient2[,1] Age1[,1]
##        <dbl> <dbl>           <dbl>           <dbl>    <dbl>
## 1          3    47          -0.420          -0.582   -0.357
## 2          3    46          -0.420          -0.582   -0.457
## 3          5    47           1.02            1.42    -0.357
## 4          3    46          -0.420          -0.582   -0.457
## 5          4    53           0.301           0.418    0.243
## 6          4    52           0.301           0.418    0.143
\end{verbatim}

Question 2b
\#\# Data Wrangle

\begin{verbatim}
##        V1          
##  Min.   :-0.70917  
##  1st Qu.:-0.46046  
##  Median :-0.21175  
##  Mean   :-0.00897  
##  3rd Qu.: 0.16131  
##  Max.   : 9.23926  
##  NA's   :57
\end{verbatim}

\begin{verbatim}
## 2018 2019 
## 2063    0
\end{verbatim}

\begin{verbatim}
## # A tibble: 2,063 x 3
##    Hours.Exercise Wave  Hours.Exercise44[,1]
##             <dbl> <fct>                <dbl>
##  1           24   2018                2.28  
##  2           10   2018                0.534 
##  3            6   2018                0.0370
##  4            0   2018               -0.709 
##  5            7   2018                0.161 
##  6            1   2018               -0.585 
##  7            1   2018               -0.585 
##  8            3   2018               -0.336 
##  9            8   2018                0.286 
## 10            5.5 2018               -0.0252
## # ... with 2,053 more rows
\end{verbatim}

Question 3
\#\# Working with dates

\begin{verbatim}
## # A tibble: 607 x 3
##    day            n Year 
##    <date>     <int> <fct>
##  1 2018-06-21   112 2018 
##  2 2018-06-22    93 2018 
##  3 2018-06-24    80 2018 
##  4 2018-06-20    67 2018 
##  5 2018-06-23    59 2018 
##  6 2018-06-26    58 2018 
##  7 2019-12-03    54 2019 
##  8 2018-06-25    52 2018 
##  9 2019-10-04    47 2019 
## 10 2019-12-02    46 2019 
## # ... with 597 more rows
\end{verbatim}

Maxmimum for 2019: 54
Maximum for 2018: 112 based from the above summary and pulling the relevant dates out from it.

Question 4
\#\# Calculating dates and creating summaries

How many days are there between the date with the highest number of responses and the date with the second highest number of responses? Pulling the relevant data from the above summary: the answer is 1 day between these dates.

\emph{Bonus}: Calculate difference between the number of responses on the highest response date and second highest response date? See below:

\begin{verbatim}
## [1] 19
\end{verbatim}

Question 5
\#\# Working with date intervals

\begin{verbatim}
## [1] 302.8338
\end{verbatim}

Question 6
\#\# Create an ordered factor from numerical data

\begin{verbatim}
## 
##      Non-attendance Moderate attendance Frequent attendance                <NA> 
##                3366                 493                 158                 109
\end{verbatim}

*Make sure to re-level the factor so that the ordinal ranking moves from lowest to highest - because the factors are already leveled correctly, no re-leveling is required -
re-leveling could otherwise be done with the re-level function

Question 7a
\#\# Make a summary table

\begin{table}

\caption{\label{tab:kablekable}Average Number of Hours of Sleep by Month}
\centering
\begin{tabu} to \linewidth {>{\raggedleft}X>{\raggedleft}X>{\raggedleft}X>{\raggedleft}X>{\raggedleft}X>{\raggedleft}X>{\raggedleft}X>{\raggedleft}X>{\raggedleft}X>{\raggedleft}X>{\raggedleft}X>{\raggedleft}X}
\hline
Jan & Feb & Mar & Apr & May & Jun & Jul & Aug & Sep & Oct & Nov & Dec\\
\hline
6.810909 & 6.666667 & 6.792891 & 7.279487 & 7.064935 & 6.927352 & 6.957453 & 6.976087 & 6.666667 & 6.912844 & 6.940741 & 6.872973\\
\hline
\end{tabu}
\end{table}

Question 7b
\#\# Make a summary graph

\includegraphics{Papaja_files/figure-latex/Parissavinglives-1.pdf}
Comment:

Let me have a think!

Briefly explain why some intervals are wider than others.

Question 8
\#\# Correlation graph

\begin{verbatim}
## Parameter          | FeelNervous | FeelWorthless | EverythingIsEffort | FeelRestless | FeelDepressed
## ----------------------------------------------------------------------------------------------------
## FeelHopeless       |    -0.29*** |          0.02 |           -0.22*** |     -0.33*** |       0.06***
## FeelDepressed      |    -0.26*** |     -6.78e-03 |           -0.17*** |     -0.30*** |              
## FeelRestless       |    -0.14*** |      -0.29*** |           -0.22*** |              |              
## EverythingIsEffort |    -0.30*** |      -0.21*** |                    |              |              
## FeelWorthless      |    -0.21*** |               |                    |              |
\end{verbatim}

\begin{verbatim}
## # Correlation Matrix (pearson-method)
## 
## Parameter          | FeelNervous | FeelWorthless | EverythingIsEffort | FeelRestless | FeelDepressed
## ----------------------------------------------------------------------------------------------------
## FeelHopeless       |     0.43*** |       0.65*** |            0.52*** |      0.43*** |       0.66***
## FeelDepressed      |     0.38*** |       0.67*** |            0.49*** |      0.38*** |              
## FeelRestless       |     0.46*** |       0.39*** |            0.46*** |              |              
## EverythingIsEffort |     0.42*** |       0.47*** |                    |              |              
## FeelWorthless      |     0.40*** |               |                    |              |              
## 
## p-value adjustment method: Holm (1979)
\end{verbatim}

\emph{What do you find most interesting about this plot?} The strong correlations present between feeling worthless, depressed, and hopeless.

**Discuss further, note the plot doesn't show me everything that's going on here people halp!

Can't see the strength of the ``FeelWorthless'' to ``FeelDepressed'' correlation

Question 9
\#\# Create a blank papaja report

Question 10
\#\# Patchwork

Use the patchwork library to create a figure with two plots on top of each other. Use the tag\_levels function to index each of the two plots. The graphs should describe some dimension of the truncated nz dataset.

\hypertarget{results}{%
\section{Results}\label{results}}

\hypertarget{discussion}{%
\section{Discussion}\label{discussion}}

\newpage

\hypertarget{references}{%
\section{References}\label{references}}

\begingroup
\setlength{\parindent}{-0.5in}
\setlength{\leftskip}{0.5in}

\hypertarget{refs}{}
\begin{CSLReferences}{1}{0}
\leavevmode\hypertarget{ref-R-papaja}{}%
Aust, F., \& Barth, M. (2020). \emph{{papaja}: {Prepare} reproducible {APA} journal articles with {R Markdown}}. Retrieved from \url{https://github.com/crsh/papaja}

\leavevmode\hypertarget{ref-R-tinylabels}{}%
Barth, M. (2021). \emph{{tinylabels}: Lightweight variable labels}. Retrieved from \url{https://github.com/mariusbarth/tinylabels}

\leavevmode\hypertarget{ref-R-base}{}%
R Core Team. (2021). \emph{R: A language and environment for statistical computing}. Vienna, Austria: R Foundation for Statistical Computing. Retrieved from \url{https://www.R-project.org/}

\end{CSLReferences}

\endgroup


\end{document}
